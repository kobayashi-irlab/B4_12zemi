%===================================================================
%   卒業論文・修士論文 TeX テンプレート
% ===================================================================

% ===== ドキュメントクラスの指定 ==============================
\documentclass{ozakilab-thesis}

% % === 追加パッケージ =================================
% \usepackage{}


% ===== メタデータ ====================================
\academicyear{令和 7 年度}
\title{マニピュレータのなぞり動作とミリ波レーダセンサを用いた構造物内部の可視化}
\university{宇都宮大学}
\department{基盤工学科}
\major{機械システム工学コース}
\author{小林 暖弥}
\degreetype{卒業論文}
\studentid{222268A}
\supervisor{
  尾崎 功一 教授\\
  ミヤグスク レナート 准教授\\
  田畑 研太 助教}
\submissiondate{2025年11月25日}

% ===== 参考文献データベースの指定 ============================
\addbibresource{references.bib}


% ===== 文書の開始 ====================================
\begin{document}

% ----- 表紙 ----
\maketitle

% ----- 目次 ----
\frontmatter
\pagestyle{headings}
\setcounter{page}{1}

\tableofcontents
\clearpage

\listoffigures
\clearpage

\listoftables
\clearpage

% ----- 本文 -----
\mainmatter
\pagestyle{headings}
\setcounter{page}{1}

\chapter{緒言}
\label{cha:introduction}
\graphicspath{{figures/01_introduction}}


\section{研究の背景}
\label{sec:background}

研究の背景を書きます.
あああああああああああああああああああああああああああああああああああああああああああああああああああああああああああああああああああああああああああああああああああああああああああああああああああああああああああああああああああああああああああああああああああああああああああああああああああああああああああああああああああああああああああああああああああああああああああああああああああああああああああああああああああああああああああああああああああああああああああああああああああああああああああああああああああああああああああああああああああああああああああああああああああああああああああああああああああああああああああああああああああああああああああああああああああああああああああああああああああああああああああああああああああああああああああああああああああああああああああああ.

段落は空行でつくります.
いいいいいいいいいいいいいいいいいいいいいいいいいいいいいいいいいいいいいいいいいいいいいいいいいいいいいいいいいいいいいいいいいいいいいいいいいいいいいいいいいいいいいいいいいいいいいいいいいいいいいいいいいいいいいいいいいいいいいいいいいいいいいいいいいいいいいいいいいいいいいいいいいいいいいいいいいいいいいいいいいいいいいいいいいいいいいいいいいいいいいいいいいいいいいいいいいいいいいいいいいいいいいいいいいいいいいいいいいいいいいいいいいいいいいいいいいいいいいいいいいいいいいいいいいいいいいいいいいいいいいいいいいいいいいいいいいいいいいいいいいいいいいいいいいいいいいいいいいいいいいいいいいいいいいいいいいいいいいいいいいいいいいいいいいいいいいいいいいいいいいいいいいいいいいいいいいいいいいいいいいいいいいいいいいいいいいいいいいいいいいいいいいいいいいいいいいいいいいいいいいいいいいいいいいいいいいいいいいいいいいいいいいいいいいいいいいいいいい.


\section{従来研究と問題点}
\label{sec:problem_statement}

従来の課題を書きます.
ううううううううううううううううううううううううううううううううううううううううううううううううううううううううううううううううううううううううううううううううううううううううううううううううううううううううううううううううううううううううううううううううううううううううううううううううううううううううううううううううううううううううううううううううううううううううううううううううううううううううううううううううううううううううううううううううううううううううううううううううううううううううううううううううううううううううううううううううううううううううううううううううううううううううううううううううううううううううううううううううううううううううううううううううううううううううううううううううううううううううううううううううううううううううううううううううううううううううううううう.


\section{本研究の目的}
\label{sec:objective}
研究目的を書きます.
えええええええええええええええええええええええええええええええええええええええええええええええええええええええええええええええええええええええええええええええええええええええええええええええええええええええええええええええええええええええええええええええええええええええええええええええええええええええええええええええええええええええええええええええええええええええええええええええええええええええええええええええええええええええええええええええええええええええええええええええええええええええええええええええええええええええええええええええええええええええええええええええええええええええええええええええええええええええええええええええええええええええええええええええええええええええええええええええええええええええええええええええええええええええええええええええええええええええええええええええええええええええええええええええええええええええええええええええええええええええええええええええええええええええええええええええええええええええええええええええええええええええええええええええええええええええええええええええええええええええええええええええええええええええええええええええええええええええええええええええええええええええええええええええええええええええええええええええええええええええええええええええええええええええええええええええええええええええええええええええええええええええええええええええええええええええええええええええええええええええええええええ.

\section{アプローチと貢献}
\label{sec:approach_and_contribution}
アプローチを書きます.
おおおおおおおおおおおおおおおおおおおおおおおおおおおおおおおおおおおおおおおおおおおおおおおおおおおおおおおおおおおおおおおおおおおおおおおおおおおおおおおおおおおおおおおおおおおおおおおおおおおおおおおおおおおおおおおおおおおおおおおおおおおおおおおおおおおおおおおおおおおおおおおおおおおおおおおおおおおおおおおおおおおおおおおおおおおおおおおおおおおおおおおおおおおおおおおおおおおおおおおおおおおおおおおおおおおおおおおおおおおおおおおおおおおおおおおおおおおおおおおおおおおおおおおおおおおおおおおおおおおおおおおおおおおおおおおおおおおおおおおおおおおおおおおおおおおおおおおおおおおおおおおおおおおおおおおおおおおおおおおおおおおおおおおおおおおおおおおおおおおおおおおおおおおおおおおおおおおおおおおおおおおおおおおおおおおおおおおおおおおおおおおおおおおおおおおおおおおおおおおおおおおおおおおおおおおおおおおおおおおおおおおおおおおおおおおおおおおおおおおおおおおおおおおおおおおおおおおおおおおおおおおおおおおおおおおおおおおおおおおおおおおおおおおおおおおおおおおおおおおおおおおおおおおおおおおおおおおおおおおお.

\section{本論文の構成}
\label{sec:organization}
論文構成を書きます.
以下のように各章を参照します.
\begin{itemize}[label={}, itemsep=0.5\baselineskip]
  \item 第\ref{cha:related_work}章では,かかかかかかかかかかかかかかかかかかかかかかかかかかかかかかかかかかかかかかかかかかかかかかかかかかかかかかかかかかかかかかかかかかかかかかかかかかかかかかかかかかかかかかかかかかかかかかかかかかかかかかかかかかかかかかかかかかかかかかかかかかかかかかかかかかかかかかかかかかかかかかかかかかかかか.
  \item 第\ref{cha:hardware_design}章では,ききききききききききききききききききききききききききききききききききききききききききききききききききききききききききききききききききききききききききききききききききききききききききききききききききききききききききききききききききききききききききききききききききききききききききききききききききききき.
  \item 第\ref{cha:software_design}章では,くくくくくくくくくくくくくくくくくくくくくくくくくくくくくくくくくくくくくくくくくくくくくくくくくくくくくくくくくくくくくくくくくくくくくくくくくくくくくくくくくくくくくくくくくくくくくくくくくくくくくくくくくくくくくく.
  \item 第\ref{cha:field_experiments}章では,けけけけけけけけけけけけけけけけけけけけけけけけけけけけけけけけけけけけけけけけけけけけけけけけけけけけけけけけけけけけけけけけけけけけけけけけけけけけけけけけけけけけけけけけけけけけけけけけけけけけけけけけけけけけけけ.
  \item 第\ref{cha:simulation_experiments}章では,こここここここここここここここここここここここここここここここここここここここここここここここここここここここここここここここここここここここここここここここここここここここここここここここここここここここここここここここここここここここここここここここここここここここここここここここここここここ.
  \item 第\ref{cha:conclusion}章では,わわわわわわわわわわわわわわわわわわわわわわわわわわわわわわわわわわわわわわわわわわわわわわわわわわわわわわわわわわわわわわわわわわわわわわわわわわわわわわわわわわわわわわわわわわわわわわわわわわわわわわわわわわわわわわわわわわわわわわわわわわわわわわわわわわわわわわわわわわわわわわわわわわわわわ.
\end{itemize}

\chapter{関連研究}
\label{cha:related_work}
\graphicspath{{figures/02_related_work}}


\section{物体をなぞる操作の必要性}
\label{sec:need_for_tracing}
インフラ内部の欠陥を可視化する手法として,地中レーダ(Ground Penetrating Radar:GPR)による検査が広く用いられている.GPRは,装置を移動させながらおおよそ50MHz$\sim$4.5GHzの電磁波を物体中に伝搬させ,内部構造物からの反射波の周波数毎の時間,強度,波形を計測することでその内部を測定する\cite{gpr}.一般に,GPRを含むレーダセンサは,物体表面から遠すぎると物体中への電波の放射が弱まり,探査深度が大幅に減少するなど測定の精度が低下する\cite{地中レーダ距離}.そのため,検査を行う際にはレーダセンサを測定対象の表面に十分近接させつつ接触しないように走査させる必要がある.

\section{物体をなぞる操作に関する従来研究}
\label{sec:related_work_tracing}

Chenらは,複雑な形状の物体の塗装のため,点群データを用いてマニピュレータの軌道計画を行った\cite{点群によるなぞり}.この研究では,点群モデルを取得し,これをスライスして物体の輪郭に対する法線ベクトルを推定することで,物体をなぞるマニピュレーションを実現した.しかし,この手法では軌道計画のために点群データの取得と処理を事前に行う必要があり,リアルタイム性が重要となるモバイルマニピュレータには向かない.検査のためには,自己位置推定やナビゲーションの誤差によってUGVと壁面との距離が変化しても,リアクティブにその誤差を吸収する必要がある.

リアルタイム性を持つ操作として,水内らは近接覚センサ情報に基づいてキッチン組み込み型ロボットが“なぞり動作”を行い,未知の食器形状を把握することを可能にした\cite{なぞり動作}.この研究では,ロボットハンド表面の同一平面上に赤外線の発光・受光素子を一体化した近接覚センサ5点を埋め込み,これらの信号強度が均一かつ一定に保つことで次の3つの動作を行った.

(1)食器表面とセンサの向きを一致させる動作

(2)食器表面との間隔を調整する動作

(3)センサ面と平行に移動する動作

\noindent
これらによって実現される“なぞり動作”を応用することで,レーダセンサを物体表面に対して十分に近接させながら走査させるマニピュレーションが達成できると考えられる.

\section{まとめ}
\label{sec:positioning}
本章では,まずGPRを含むレーダセンサの特性について述べ,それらを測定対象の表面に十分近接させつつ接触しないように走査させるため,物体をなぞる操作が必要であることを示した.また,なぞり動作に関する従来研究を紹介し,リアルタイム性を持ったマニピュレーションが求められることから,複数の赤外線近接覚センサを用いた“なぞり動作”が検査に有効であると示した.これを踏まえ,本研究では,なぞり動作によってレーダセンサを走査し,構造物内部の物体を可視化するシステムを実装した.次章では,その詳細について述べる.

\chapter{なぞり動作とレーダセンサによる構造物内部の可視化}
\label{cha:system_impl}
\graphicspath{{figures/03_system_impl}}


\section{エンドエフェクタの設計}
\label{sec:hardware_overview}

製作したエンドエフェクタの外観を\ref{fig:ee外観}に示す.エンドエフェクタには,赤外線近接距離センサモジュール(pololu, VL6180X)が同一平面上の91mm$\times$91mmの正方形の頂点に4つと,各辺の中心に4つの計8つ設置されている.この配置は,エンドエフェクタの姿勢の変化を安定して取得しつつ,後述するセンサ情報の計算を容易にするために設計されている.なお,これらの近接覚センサはその位置の重心がエンドエフェクタの中心と一致するよう取り付けられている.また,中央部にはレーダセンサを取り付けるための支柱と穴が設けられている.近接覚センサは$0\sim255$mmの範囲で1mmの分解能で距離を測定するが,各モジュールごとに得られる距離値に個体差があるため,使用する際はオフセットを設定してそれぞれの値を均一化する.

\begin{figure}[tb]
  \centering
  \includegraphics[width=.8\linewidth]{ee外観.pdf}
  \caption{Appearance of the end effector}
  \label{fig:ee外観}
\end{figure}

\section{走査するレーダセンサの選定}
\label{sec:mechanical_design}

レーダセンサの解像度はレーダの波長に依存しているため,GPRでは5cm未満の物体が「点」として検出されたり,数cm間隔で並ぶ鉄筋が別個のオブジェクトとして認識されないなどといった問題があった\cite{gpr_point}\cite{gpr_unrecog}.一方で,波長が短く周波数が高いレーダセンサは物体中で減衰しやすく探査深度が浅くなってしまうことからインフラ等の検査にはあまり使用されない.しかし,本研究ではより高解像度に物体中の構造を可視化することを目指してミリ波レーダを用いる.中でも,活用の際の法律的なハードルが低いことから60GHz帯のレーダを使用する.ミリ波レーダは,acconeer社製のA121レーダセンサチップを搭載した評価キットXE121と,コネクタボードXC120を用いる.



\section{なぞり動作のアルゴリズム}
\label{sec:hardware_evaluation}

なぞり動作は,エンドエフェクタを取り付けた7自由度協働ロボットアーム(UFactory, xArm7)によって行う.エンドエフェクタには\ref{fig:ee_coordinate}に示すように座標系が定義されており,これにしたがって近接覚センサの情報を計算して速度および角速度を与える.なお,速度と角速度は近接覚センサの誤差による振動を抑えるためにPD制御されている.

以上の計算に用いるパラメータを,\ref{fig:計算に用いる文字}に示す.エンドエフェクタの面上にN個の近接覚センサが分布しており,その面の法線ベクトル$\boldsymbol{n}$の方向を計測しているとき,エンドエフェクタ座標系における各センサの位置を$\boldsymbol{p}_i$,各センサから得られる計測距離を$s_i$とすると$(i = 1,2,\ldots,N)$,センサの位置の重心$\boldsymbol{p}_C$およびセンサの計測距離の重心$\boldsymbol{p}_{SG}$は次式で表される.
\begin{equation}
\boldsymbol{p}_C = \frac{\sum_{i=1}^N \boldsymbol{p}_i}{N}, 
\quad 
\boldsymbol{p}_{SG} = \frac{\sum_{i=1}^N s_i \boldsymbol{p}_i}{\sum_{i=1}^N s_i}
\end{equation}

\begin{figure}[tb]
    \centering
    \includegraphics[width=.8 \linewidth]{ee_coordinate.pdf}
    \caption{Endeffector coordinate system}
    \label{fig:ee_coordinate}
\end{figure}

\begin{figure}[htbp]
  \centering
  \begin{subfigure}{0.45\linewidth}
    \centering
    \includegraphics[width=\linewidth]{センサ計算説明図.pdf}
    \caption{Distribution of sensor data}
    \label{fig:センサ計算説明図}
  \end{subfigure}
  \hfill
  % 右側の図
  \begin{subfigure}{0.45\linewidth}
    \centering
    \includegraphics[width=\linewidth]{速度指令説明図.pdf}
    \caption{Center of mass, velocity, and angular velocity}
    \label{fig:速度指令説明図}
  \end{subfigure}
  \caption{Calculation symbols}
  \label{fig:計算に用いる文字}
\end{figure}

\noindent
センサ情報$s_i$に基づき,このセンサ位置の重心$\boldsymbol{p}_C$に対して物体表面のなぞり動作を行うための目標速度$\boldsymbol{v}$と目標角速度$\boldsymbol{\omega}$を与える.目標速度$\boldsymbol{v}$は次式となる.

\begin{equation}
\boldsymbol{v} = \boldsymbol{v}_{trace} + \boldsymbol{v}_{ctrl}
\end{equation}

\noindent
ここで,$\boldsymbol{v}_{trace}$は物体に沿ってセンサ面に平行な方向にエンドエフェクタを動かすための速度の指令値である.また,$\boldsymbol{v}_{ctrl}$は物体表面とセンサ面の距離を一定に保つための速度であり,PD制御されている.これらは,近接覚センサが計測した距離の最小値${s}_{min}$と,センサ面と物体表面の目標距離${s}_{target}$の差$s_e$に基づいて次のように決定される.なお,$k_{vP}$と$k_{vD}$はそれぞれ比例ゲインと微分ゲインである.
\begin{align}
{s}_{min} &= \max_i {s}_i
\\
{s_e} &= {s}_{min} - {s}_{target}
\\
\boldsymbol{v}_{ctrl} &= k_{vP}s_e\boldsymbol{n} + k_{vD}(s_e - s_{e-1})\boldsymbol{n}
\end{align}

\noindent
次に,物体表面とセンサ面を平行にするための回転速度である目標角速度$\boldsymbol{\omega}$は次式で表される.なお,$k_{\omega P}$と$k_{\omega D}$は各比例ゲインと微分ゲインである.
\begin{align}
\boldsymbol{p}_e &= \boldsymbol{p}_C - \boldsymbol{p}_{SG}
\\
\boldsymbol{\omega} &= k_{\omega P}\boldsymbol{n} \times \boldsymbol{p}_e + k_{\omega D}\boldsymbol{n} \times (\boldsymbol{p}_e - \boldsymbol{p}_{e-1})
\end{align}

\noindent
$\boldsymbol{p}_C - \boldsymbol{p}_{SG}$は計測距離の重心$\boldsymbol{p}_{SG}$から位置の重心$\boldsymbol{p}_C$に向かうセンサ面内のベクトルであるから,位置の重心$\boldsymbol{p}_C$を中心としてエンドエフェクタに$\boldsymbol{n} \times (\boldsymbol{p}_C - \boldsymbol{p}_{SG})$方向の回転をさせると$\boldsymbol{p}_{SG}$が物体表面に近づくように動き,その結果$\boldsymbol{p}_{SG}$と$\boldsymbol{p}_C$が一致する動きをとることが期待される.これにより,なぞり動作が実現される.



\section{可視化システム構成}
\label{sec:hardware_evaluation}


\chapter{ソフトウェア設計}
\label{cha:software_design}
\graphicspath{{figures/04_software_design}}


\section{カルマンフィルタの実装}
\label{sec:kalman_filter}

カルマンフィルタの疑似コードは\ref{alg:kalman_filter}です.
あああああああああああああああああああああああああああああああああああああああああああああああああああああああああああああああああああああああああああああああああああああああああああああああああああああああああああああああああああああああああああああああああああああああああああああああああああああああああああああああああああああああああああああああああああああああああああああああああああああああああああああああああああああああああああああああああああああああああああああああああああああああああああああああああああああああああああああああああああああああああああああああああああああああああああああ

\begin{algorithm}
  \caption{Kalman Filter}
  \label{alg:kalman_filter}
  \begin{algorithmic}[1]
    \State \textbf{Input:} Initial State $\mathbf{\hat{x}}_{0|0}$, Initial Covariance $\mathbf{P}_{0|0}$
    \State \textbf{Output:} Estimated State at time $k$, $\mathbf{\hat{x}}_{k|k}$
    \State
    \For{$k = 1, 2, \dots$}
    %
    \State \textbf{Prediction Step}
    %
    \State \quad $\mathbf{\hat{x}}_{k|k-1} \gets \mathbf{A} \mathbf{\hat{x}}_{k-1|k-1} + \mathbf{B} \mathbf{u}_k$
    \Comment{Project the state ahead}
    \State \quad $\mathbf{P}_{k|k-1} \gets \mathbf{A} \mathbf{P}_{k-1|k-1} \mathbf{A}^{\top} + \mathbf{Q}$
    \Comment{Project the error covariance ahead}
    \State 
    %
    \State \textbf{Update Step}
    %
    \State \quad $\mathbf{K}_k \gets \mathbf{P}_{k|k-1} \mathbf{C}^{\top} (\mathbf{C} \mathbf{P}_{k|k-1} \mathbf{C}^{\top} + \mathbf{R})^{-1}$
    \Comment{Compute the Kalman gain}
    \State \quad $\mathbf{\hat{x}}_{k|k} \gets \mathbf{\hat{x}}_{k|k-1} + \mathbf{K}_k (\mathbf{y}_k - \mathbf{C} \mathbf{\hat{x}}_{k|k-1})$
    \Comment{Update the state estimate}
    \State \quad $\mathbf{P}_{k|k} \gets (\mathbf{I} - \mathbf{K}_k \mathbf{C}) \mathbf{P}_{k|k-1}$
    \Comment{Update the error covariance}
    \EndFor
  \end{algorithmic}
\end{algorithm}

\section{ソフトウェア評価}
\label{sec:software_evaluation}
ソフトウェアの評価をします.
いいいいいいいいいいいいいいいいいいいいいいいいいいいいいいいいいいいいいいいいいいいいいいいいいいいいいいいいいいいいいいいいいいいいいいいいいいいいいいいいいいいいいいいいいいいいいいいいいいいいいいいいいいいいいいいいいいいいいいいいいいいいいいいいいいいいいいいいいいいいいいいいいいいいいい

\chapter{実環境におけるシステム評価}
\label{cha:field_experiments}
\graphicspath{{figures/05_field_experiments}}


\section{実験目的}
\label{sec:exp_purpose}

\eqref{eq:normal_distribution}はガウス分布です.
\begin{equation}
  f(x \mid \mu, \sigma^2) = \frac{1}{\sqrt{2\pi\sigma^2}} \exp\left(-\frac{(x-\mu)^2}{2\sigma^2}\right)
  \label{eq:normal_distribution}
\end{equation}



\section{実験環境}
\label{sec:exp_environment}

実験環境を説明します.


\section{実験結果}
\label{sec:exp_results}

ロボットのオドメトリは\eqref{eq:odometry}で計算されます.
次の位置姿勢 $(x_{k+1}, y_{k+1}, \theta_{k+1})$ は,現在の位置姿勢 $(x_k, y_k, \theta_k)$ と,微小時間 $\Delta t$ の間の並進速度 $v$ および角速度 $\omega$ を使って更新されます.

\begin{equation}
  \begin{cases}
    x_{k+1} = x_k + v \Delta t \cos(\theta_k + \frac{\omega \Delta t}{2}) \\
    y_{k+1} = y_k + v \Delta t \sin(\theta_k + \frac{\omega \Delta t}{2}) \\
    \theta_{k+1} = \theta_k + \omega \Delta t
  \end{cases}
  \label{eq:odometry}
\end{equation}

\subsection{現在の位置姿勢}
$(x_k, y_k, \theta_k)$ は現在の位置姿勢です.

\subsection{次の位置姿勢}
$(x_{k+1}, y_{k+1}, \theta_{k+1})$ 次の位置姿勢です.

\subsection{更新}
微小時間 $\Delta t$ の間の並進速度 $v$ および角速度 $\omega$ を使って更新されます.

\section{考察}
\label{sec:exp_discussion}

考察します.

\chapter{シミュレーション評価}
\label{cha:simulation_experiments}
\graphicspath{{figures/05_simulation_experiments}}


\section{シミュレーション環境の構築}
\label{sec:sim_environment}

環境をつくりました.
あああああああああああああああああああああああああああああああああああああああああああああああああああああああああああああああああああああああああああああああああああああああああああああああああああああああああああああああああああああああああああ


\section{実験概要}
\label{sec:sim_exp}

シミュレーションします.
いいいいいいいいいいいいいいいいいいいいいいいいいいいいいいいいいいいいいいいいいいいいいいいいいいいいいいいいいいいいいいいいいいいいいいいいいいいいいいいいいいいいいいいいいいいいいいいいいいいいいいいいいいいいいいいいいいいいいいいいいいいいいいいいい


\section{考察}

考察を\ref{tab:performance_comparison}にまとめます.
うううううううううううううううううううううううううううううううううううううううううううううううううううううううううううううううううううううううううううううううううううううううううううううううううううううううううううううううううううううううううううううううう

\label{sec:sim_discussion}
\begin{table}[htb]
  \centering
  \caption{性能比較}
  \label{tab:performance_comparison}
  \begin{tabular}{lccc}
    \toprule
    項目 & 提案手法 & 従来手法 A & 従来手法 B \\
    \midrule
    精度 (\%) & 95.2 & 92.1 & 88.5 \\
    処理時間 (ms) & 12.5 & 18.0 & 10.2 \\
    消費電力 (W) & 3.1 & 3.5 & 2.9 \\
    \bottomrule
  \end{tabular}
\end{table}

\chapter{結言}
\label{cha:conclusion}
\graphicspath{figures/07_conclusion}


\section{結論}
\label{sec:summary}

結論を述べます.


\section{課題}

課題を述べます.


\section{展望}
\label{sec:future_work}

展望を述べます.
\clearpage

% ----- 謝辞 -----
\pagestyle{plain}
\chapter*{謝辞}
\label{cha:acknowledgements}
\addcontentsline{toc}{chapter}{謝辞} % 目次にも表示する場合

先生に感謝します.
ああああああああああああああああああああああああああああああああああああああああああああああああああああああああああああああああああああああああああああああああああああああああああああああああああああああああああああああああああああああああああああああああああああ

\vspace{1\baselineskip}

共同研究者に感謝します.
いいいいいいいいいいいいいいいいいいいいいいいいいいいいいいいいいいいいいいいいいいいいいいいいいいいいいいいいいいいいいいいいいいいいいいいいいいいいいいいいいいいいいいいいいいいいいいいいいいいいいいいいいいいいいいいいいいいいいいいいいいいいいいいいいいいい

\vspace{1\baselineskip}

研究室メンバーに感謝します.
うううううううううううううううううううううううううううううううううううううううううううううううううううううううううううううううううううううううううううううううううううううううううううううううううううううううううううううううううううううううううううううううううううう

\vspace{1\baselineskip}

家族に感謝します.
ええええええええええええええええええええええええええええええええええええええええええええええええええええええええええええええええええええええええええええええええええええええええええええええええええええええええええええええええええええええええええええええええええええ

\vspace{1\baselineskip}

結びます.
おおおおおおおおおおおおおおおおおおおおおおおおおおおおおおおおおおおおおおおおおおおおおおおおおおおおおおおおおおおおおおおおおおおおおおおおおおおおおおおおおおおおおおおおおおおおおおおおおおおおおおおおおおおおおおおおおおおおおおおおおおおおおおおおおおおお

\vspace{1\baselineskip}

\begin{flushright}
  著者氏名
\end{flushright}

\clearpage

% ----- 参考文献 -----
\pagestyle{headings}
% \printbibliography[heading=bibintoc, title={参考文献}]
\printbibliography

% ----- 付録 -----
\appendix
\pagestyle{headings}
\chapter{詳細な電子回路図}
\label{apx:schematics}
\graphicspath{{figures/a_schematics}}


電子回路図を\ref{fig:full_schematic}に示します.
あああああああああああああああああああああああああああああああああああああああああああああああああああああああああああああああああああああああああああああああああああああああああああああああああああああああああああああああああああああああああああ

\begin{figure}[htbp]
  \centering
  \includegraphics[width=\linewidth]{real.png}
  \caption{システム全体の回路図}
  \label{fig:full_schematic}
\end{figure}

\chapter{主要なプログラムのソースコード}
\label{apx:source_code}
\graphicspath{figures/b_source_code}

\section{C++}
C++ソースコードを\ref{lst:hw_cpp}に示します.
あああああああああああああああああああああああああああああああああああああああああああああああああああああああああああああああああああ.

\begin{lstlisting}[
  language=C++,
  caption={Hello World Program in C++},
  label={lst:hw_cpp}
]
// C++ source code here.
#include <iostream>

int main() {
  std::cout << "Hello, World!" << std::endl;
  return 0;
}
\end{lstlisting}


\section{Python}
Pythonソースコードを\ref{lst:hw_python}に示します.
いいいいいいいいいいいいいいいいいいいいいいいいいいいいいいいいいいいいいいいいいいいいいいいいいいいいいいいいいいいいいいいいいいいい.
\begin{lstlisting}[
  language=Python,
  caption={Hello World Program in Python},
  label={lst:hw_python}
]
# Python source code here.
# The simplest way to print a line in Python:

def main():
  print("Hello, World!")

if __name__ == "__main__":
  main()
\end{lstlisting}

\chapter{追加実験}
\label{apx:additional_experiments}
\graphicspath{figures/c_additional_experiments}

\section{実験A}
ああああああああああああああああああああああああああああああああああああああああああああああああああああああああああああああああああああああああああああ

\section{実験B}
いいいいいいいいいいいいいいいいいいいいいいいいいいいいいいいいいいいいいいいいいいいいいいいいいいいいいいいいいいいいいいいいいいいいいいいいいいいいいいいいいいいいいいいいいいいいいいいいいいいいいいい

\end{document}