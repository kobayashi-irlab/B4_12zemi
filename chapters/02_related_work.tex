\chapter{関連研究}
\label{cha:related_work}
\graphicspath{{figures/02_related_work}}


\section{物体をなぞる操作の必要性}
\label{sec:need_for_tracing}
インフラ内部の欠陥を可視化する手法として,地中レーダ(Ground Penetrating Radar:GPR)による検査が広く用いられている.GPRは,装置を移動させながらおおよそ50MHz$\sim$4.5GHzの電磁波を物体中に伝搬させ,内部構造物からの反射波の周波数毎の時間,強度,波形を計測することでその内部を測定する\cite{gpr}.一般に,GPRを含むレーダセンサは,物体表面から遠すぎると物体中への電波の放射が弱まり,探査深度が大幅に減少するなど測定の精度が低下する\cite{地中レーダ距離}.そのため,検査を行う際にはレーダセンサを測定対象の表面に十分近接させつつ接触しないように走査させる必要がある.

\section{物体をなぞる操作に関する従来研究}
\label{sec:related_work_tracing}

Chenらは,複雑な形状の物体の塗装のため,点群データを用いてマニピュレータの軌道計画を行った\cite{点群によるなぞり}.この研究では,点群モデルを取得し,これをスライスして物体の輪郭に対する法線ベクトルを推定することで,物体をなぞるマニピュレーションを実現した.しかし,この手法では軌道計画のために点群データの取得と処理を事前に行う必要があり,リアルタイム性が重要となるモバイルマニピュレータには向かない.検査のためには,自己位置推定やナビゲーションの誤差によってUGVと壁面との距離が変化しても,リアクティブにその誤差を吸収する必要がある.

リアルタイム性を持つ操作として,水内らは近接覚センサ情報に基づいてキッチン組み込み型ロボットが“なぞり動作”を行い,未知の食器形状を把握することを可能にした\cite{なぞり動作}.この研究では,ロボットハンド表面の同一平面上に赤外線の発光・受光素子を一体化した近接覚センサ5点を埋め込み,これらの信号強度が均一かつ一定に保つことで次の3つの動作を行った.

(1)食器表面とセンサの向きを一致させる動作

(2)食器表面との間隔を調整する動作

(3)センサ面と平行に移動する動作

\noindent
これらによって実現される“なぞり動作”を応用することで,レーダセンサを物体表面に対して十分に近接させながら走査させるマニピュレーションが達成できると考えられる.

\section{まとめ}
\label{sec:positioning}
本章では,まずGPRを含むレーダセンサの特性について述べ,それらを測定対象の表面に十分近接させつつ接触しないように走査させるため,物体をなぞる操作が必要であることを示した.また,なぞり動作に関する従来研究を紹介し,リアルタイム性を持ったマニピュレーションが求められることから,複数の赤外線近接覚センサを用いた“なぞり動作”が検査に有効であると示した.これを踏まえ,本研究では,なぞり動作によってレーダセンサを走査し,構造物内部の物体を可視化するシステムを実装した.次章では,その詳細について述べる.