\chapter{緒言}
\label{cha:introduction}
\graphicspath{{figures/01_introduction}}


\section{研究の背景}
\label{sec:background}

日本のインフラは高度成長期以降に集中的に整備されたことから,今後20年間で建設後50年以上経過する施設の割合が加速度的に高くなると見込まれており,一斉に老朽化する社会資本を維持管理していくことが課題となっている\cite{老朽化}.一方で,建設業就業者はピーク時から200万人減少,就業者の3割以上が55歳以上と高齢化も進み,労働力不足が深刻化している\cite{高齢化}.こうしたことから,インフラメンテナンスの自動化,省人化が進められている.特に非破壊検査の領域では,従来の手作業に比べて高速かつ高精度な点検が可能となることや,データの収集と解析を効率的に行えるようになることから,ロボットの導入が進められている.

非破壊検査を行うロボットの主な形態として,無人航空機(UAV),地上無人車両(UGV)が挙げられる.UAVは,機動性が高く人間がアクセスしにくい場所でもデータを収集できるが,風などの外部環境に左右されやすいほか,ペイロードが小さく搭載できる機器や活動時間が制限されるため,一度のフライトでは多様な検査が行えない.また,UGVはペイロードが大きく様々なセンサを搭載できるが,検査の対象は一般的に接地面に限られる\cite{robot_example}.そこで,こうした弱点を補完し,一台で様々な検査が可能となるロボットの形態として,モバイルマニピュレータが注目されている.

モバイルマニピュレータは,ロボットマニピュレータをUGVなどのモバイルプラットフォームに搭載したシステムのことである\cite{mobile_manipulator}.これを非破壊検査に応用すると,マニピュレータが壁面を,UGVが接地面を検査することによって,精密で広域な検査が可能となる.また,マニピュレータのエンドエフェクタを付け替えることにより多様なセンサによるマルチな検査を行うことができる.

\section{本研究の目的}
\label{sec:objective}
モバイルマニピュレータが検査を行う際,マニピュレータはUGVの動作に合わせて適切に検査対象面をなぞりながらセンサを走査させる必要があるため,その制御が非常に重要である.本研究では,そのなぞり動作を近接覚センサを用いたマニピュレーションによって実現する.加えて,水道管や空洞,埋没物の非破壊検査を念頭に置き,ミリ波レーダセンサを走査して構造物内部を可視化するシステムを実現する.

\section{本論文の構成}
\label{sec:organization}
本論文の構成を以下に述べる.
\begin{itemize}[label={}, itemsep=0.5\baselineskip]
  \item 本章では,研究背景および研究目的について述べた.
  \item 第\ref{cha:related_work}章では,物体なぞる操作の必要性と,なぞり動作の関連研究について述べる.
  \item 第\ref{cha:system_impl}章では,ききききききききききききききききききききききききききききききききききききききききききききききききききききききききききききききききききききききききききききききききききききききききききききききききききききききききききききききききききききききききききききききききききききききききききききききききききききき.
  \item 第\ref{cha:software_design}章では,くくくくくくくくくくくくくくくくくくくくくくくくくくくくくくくくくくくくくくくくくくくくくくくくくくくくくくくくくくくくくくくくくくくくくくくくくくくくくくくくくくくくくくくくくくくくくくくくくくくくくくくくくくくくくく.
  \item 第\ref{cha:field_experiments}章では,けけけけけけけけけけけけけけけけけけけけけけけけけけけけけけけけけけけけけけけけけけけけけけけけけけけけけけけけけけけけけけけけけけけけけけけけけけけけけけけけけけけけけけけけけけけけけけけけけけけけけけけけけけけけけけ.
  \item 第\ref{cha:simulation_experiments}章では,こここここここここここここここここここここここここここここここここここここここここここここここここここここここここここここここここここここここここここここここここここここここここここここここここここここここここここここここここここここここここここここここここここここここここここここここここここここ.
  \item 第\ref{cha:conclusion}章では,わわわわわわわわわわわわわわわわわわわわわわわわわわわわわわわわわわわわわわわわわわわわわわわわわわわわわわわわわわわわわわわわわわわわわわわわわわわわわわわわわわわわわわわわわわわわわわわわわわわわわわわわわわわわわわわわわわわわわわわわわわわわわわわわわわわわわわわわわわわわわわわわわわわわわ.
\end{itemize}